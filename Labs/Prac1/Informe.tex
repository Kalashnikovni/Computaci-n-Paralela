\nonstopmode
\documentclass[a4paper,10pt]{article}
\usepackage[utf8]{inputenc}
\usepackage{textcomp}
\usepackage[english]{babel}
\usepackage{amsmath}
\usepackage{amssymb}
\usepackage[margin=1in]{geometry}
\usepackage{indentfirst}
\usepackage{graphicx}

\author{
    Marzorati Denise \\
    \texttt{M-6219/7}
}

\date{
    20 de abril de 2018
}

\title{
    \begin{center}
\includegraphics[scale=1]{logo-unr.png}
\end{center}
\Huge \textsc{{\bfseries T}rabajo {\bfseries P}ráctico 1: Optimización secuencial} \\
    \large \textsc{Computación paralela} \\
}

\begin{document}

\bigskip
\bigskip
\bigskip

\maketitle

\thispagestyle{empty}

\begin{center}
\large \bf Docentes de la materia
\end{center}

\begin{center}
Nicolás Wolovick

Carlos Bederián
\end{center}

\newpage{}
\normalsize

\section*{Características del hardware y del software}

\begin{itemize}
  % Obtenemos con: cat /proc/cpuinfo 
  \item CPU: Intel Core i7-3632QM @ 2.20GHz.
  % Obtenemos con: sudo lshw -short -C memory
  \item Memoria: 8GB, DDR3, 2 canales. 
  \item L1 cache: 256KB
  \item L2 cache: 1MB
  \item L3 cache: 3MB
  % Obtenemos con: gcc -v
  \item Compilador: GCC 7.1.0.
  % Obtenemos con: uname -a (el kernel)
  % lsb_release -a (SO)
  \item Sistema operativo: Ubuntu 16.04.4, x86\_64.
\end{itemize}

\section*{Metodología de trabajo}

Para comenzar a atacar cada problema lo primero que se hará es realizar algo de profiling utilizando la
herramienta perf, para poder identificar a dónde debe dirigirse el mayor esfuerzo en optimizar. 
Se hará un breve resumen con las optimizaciones realizadas que dieron resultados significativos tras
repetir varias veces la ejecución del programa.

\newpage{}
\section*{IntegralImage}

Para obtener la versión más rápida posible se ha compilado el código con:

gcc integralimage.c -o integral -O2 -Wall -Wextra -std=c99 -msse4 -funroll-loops -lm -lgomp.

Al hacer profiling se pudo observar que la función rand() consumía gran parte del tiempo de computación, por
lo que se buscó versiones más rápidas de la función, encontrando así la presentada en el código como
fastRandSSE4() que es más veloz ya que hace uso de las instrucciones SSE4, generando cuatro números
aleatorios por vez.

Otra mejora fue la de hacer un poco de unrolling en el loop donde se realizan todos los cálculos
para la generación de la imagen. 

Por otro lado, se intentó hacer algo de loop tiling, alignment y  aliasing; dichas técnicas no aportaron 
mejoras al rendimiento del programa. Todos los intentos de mejoras algorítmicas fueron fallidos, incluso
utilizando librerías eficientes.
\\
\\
Para FRAMES = 300 los resultados han sido los siguientes:
\begin{itemize}
  \item Avg IPC: 2.61
  \item Avg percentage cache misses: 16.16\%
  \item Avg time: 0.991633s 
  \item Avg percentage stalled cycles, frontend: 18.61\%
\end{itemize}
\\
Para FRAMES = 3000 los resultados han sido los siguientes:
\begin{itemize}
  \item Avg IPC: 2.69
  \item Avg percentage cache misses: 12.02\%
  \item Avg time: 9.420287s 
  \item Avg percentage stalled cycles, frontend: 17.61\%
\end{itemize}
\\
Para FRAMES = 30000 los resultados han sido los siguientes:
\begin{itemize}
  \item Avg IPC: 2.68
  \item Avg percentage cache misses: 12.61\%
  \item Avg time: 95.767781s 
  \item Avg percentage stalled cycles, frontend: 17.89\%
\end{itemize}

\\
Con las optimizaciones realizadas se ha podido observar que ahora el tiempo que el programa tiene
de ejecución crece linealmente con respecto al tamaño del mismo, en este caso dado por el valor
FRAMES.  

\end{document}
