\nonstopmode
\documentclass[a4paper,10pt]{article}
\usepackage[utf8]{inputenc}
\usepackage{textcomp}
\usepackage[english]{babel}
\usepackage{amsmath}
\usepackage{amssymb}
\usepackage[margin=1in]{geometry}
\usepackage{indentfirst}
\usepackage{graphicx}

\author{
    Marzorati Denise \\
    \texttt{M-6219/7}
}

\date{
    10 de mayo de 2018
}

\title{
    \begin{center}
\includegraphics[scale=1]{logo-unr.png}
\end{center}
\Huge \textsc{{\bfseries T}rabajo {\bfseries P}ráctico 1: Optimización secuencial} \\
    \large \textsc{Computación paralela} \\
}

\begin{document}

\bigskip
\bigskip
\bigskip

\maketitle

\thispagestyle{empty}

\begin{center}
\large \bf Docentes de la materia
\end{center}

\begin{center}
Nicolás Wolovick

Carlos Bederián
\end{center}

\newpage{}
\normalsize

\section*{Características del hardware y del software}

\begin{itemize}
  % Obtenemos con: cat /proc/cpuinfo 
  \item CPU: Intel Core i7-3632QM @ 2.20GHz.
  % Obtenemos con: sudo lshw -short -C memory
  \item Memoria: 8GB, DDR3, 2 canales. 
  \item L1 cache: 256KB
  \item L2 cache: 1MB
  \item L3 cache: 3MB
  % Obtenemos con: gcc -v
  \item Compiladores: GCC 7.1.0, Clang 5.0, PGI, AOCC (compilador de AMD).
  % Obtenemos con: uname -a (el kernel)
  % lsb_release -a (SO)
  \item Sistema operativo: Ubuntu 16.04.4, x86\_64.
\end{itemize}

\newpage{}
\section*{IntegralImage}

Para obtener la versión más rápida posible se ha compilado el código con:

clang-5.0 integralimage.c -o integral -O2 -march=ivybridge -funroll-loops -Wall -Wextra -std=c99 -lm -lgomp.

Una de las primeras cosas que se hizo fue la de probar usando distintos compiladores. Clang fue
con el que se obtuvieron los mejores resultados, con GCC y AOCC devolviendo valores ligeramente
inferiores a los del mencionado compilador. Por otro lado, PGI tuvo un rendimiento pésimo,
puesto que para FRAMES=300 la ejecución del programa tomó alrededor de ciento cincuenta segundos.

Para lograr autovectorización se escribieron varias versiones de la función "integral\_image",
usando las técnicas mostradas en la clase correspondiente al tema. Es decir, el uso de restrict,
se indicó el alineamiento de las variables en memoria, se redujeron los múltiples loops a un
único loop. Sin embargo nada de esto logró que el compilador autovectorizara.

Por el motivo antes mencionado se prosiguió intentando vectorizar la función a mano, pero cada
uno de los intentos bajó el rendimiento del programa notablemente, así que la opción se descartó.

Como Carlos envió un mail avisando de que había un problema para usar ISPC, ni siquiera se intentó
usar el compilador.

En resumen, con respecto al trabajo anterior lo único que dio resultado fue cambiar el compilador,
que pasó de ser GCC-7 a Clang-5. Nada fue modificado del código.

Aclaración: sin haber visto el tema en las clases, para el primer trabajo se utilizó una función
generadora de números aleatorios que hace uso de SSE4, y está escrita con intrinsics, lo que
significó una mejora importante en el rendimiento del programa. 

\subsubsection*{FRAMES = 300}
\begin{table}[h]
\centering
\label{my-label}
\begin{tabular}{|l|c|c|}
\hline
                                 & \multicolumn{1}{l|}{Práctico 1} & \multicolumn{1}{l|}{Práctico 2} \\ \hline
Avg IPC                          & 2.61                            & 2.25                            \\ \hline
Avg \% cache misses              & 16.16\%                         & 14.02\%                         \\ \hline
Avg time                         & \multicolumn{1}{l|}{0.991633s}  & \multicolumn{1}{l|}{0.8807468s} \\ \hline
Avg \% stalled cycles, front end & 18.61\%                         & 21.21\%                         \\ \hline
\end{tabular}
\caption{Resultados para FRAMES = 300}
\end{table}

\subsubsection*{FRAMES = 3000}
\begin{table}[h]
\centering
\begin{tabular}{|l|c|c|}
\hline
                                 & \multicolumn{1}{l|}{Práctico 1} & \multicolumn{1}{l|}{Práctico 2} \\ \hline
Avg IPC                          & 2.69                            & 2.48                            \\ \hline
Avg \% cache misses              & 12.02\%                         & 11.69\%                         \\ \hline
Avg time                         & \multicolumn{1}{l|}{9.420287s}  & \multicolumn{1}{l|}{8.802669s}  \\ \hline
Avg \% stalled cycles, front end & 17.61\%                         & 16.67\%                         \\ \hline
\end{tabular}
\caption{Resultados para FRAMES = 3000}
\label{my-label}
\end{table}

\subsubsection*{FRAMES = 30000}
\begin{table}[h]
\centering
\begin{tabular}{|l|c|c|}
\hline
                                 & \multicolumn{1}{l|}{Práctico 1} & \multicolumn{1}{l|}{Práctico 2} \\ \hline
Avg IPC                          & 2.68                            & 2.53                            \\ \hline
Avg \% cache misses              & 12.61\%                         & 9.40\%                          \\ \hline
Avg time                         & \multicolumn{1}{l|}{95.767781s} & \multicolumn{1}{l|}{82.109516s} \\ \hline
Avg \% stalled cycles, front end & 17.89\%                         & 15.46\%                         \\ \hline
\end{tabular}
\caption{Resultados para FRAMES = 30000}
\label{my-label}
\end{table}

\\
Con las optimizaciones realizadas se ha podido observar que ahora el tiempo que el programa tiene
de ejecución crece linealmente con respecto al tamaño del mismo, en este caso dado por el valor
FRAMES.  

No han surgido ideas para potenciales mejoras en la vectorización.

\end{document}
